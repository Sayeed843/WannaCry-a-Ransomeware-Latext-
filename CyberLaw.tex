\documentclass[a4paper,10pt]{article}
\usepackage[utf8]{inputenc}

\begin{document}
\begin{titlepage}
	\centering % Centre everything on the title page
	
	\scshape % Use small caps for all text on the title page
	
	\vspace*{\baselineskip} % White space at the top of the page
	
	%------------------------------------------------
	%	Title
	%------------------------------------------------
	
	\rule{\textwidth}{1.6pt}\vspace*{-\baselineskip}\vspace*{2pt} % Thick horizontal rule
	\rule{\textwidth}{0.4pt} % Thin horizontal rule
	
	\vspace{0.75\baselineskip} % Whitespace above the title
	
	{\LARGE "WannaCry",a Ransomeware that\\ changed the status of world's \\ cyber security\\} % Title
	
	\vspace{0.75\baselineskip} % Whitespace below the title
	
	\rule{\textwidth}{0.4pt}\vspace*{-\baselineskip}\vspace{3.2pt} % Thin horizontal rule
	\rule{\textwidth}{1.6pt} % Thick horizontal rule
	
	\vspace{2\baselineskip} % Whitespace after the title block
	
	%------------------------------------------------
	%	Subtitle
	%------------------------------------------------
	
	NSA's powerful Windows hacking tools leaked online \\ $-$ CNN (April15, 2017) % Subtitle or further description
	
	\vspace*{3\baselineskip} % Whitespace under the subtitle
	
	%------------------------------------------------
	%	Editor(s)
	%------------------------------------------------
	
	
	
	\vspace{9.5\baselineskip} % Whitespace before the editors
	Created By
	\\
	{\scshape\large Sayeed Bin Mozahid \\ ID: 151-35-843 \\ } % Editor list
	{\scshape\Medium Department  of  Software Engineering \\}
	{\scshape\Medium Daffodil Intertaional University }
	
	\vspace{0.5\baselineskip} % Whitespace below the editor list
	
	\textit{ } % Editor affiliation
	
	\vfill % Whitespace between editor names and publisher logo
	
	%------------------------------------------------
	%	Publisher
	%------------------------------------------------
	
	% \plogo % Publisher logo
	
	\vspace{0.3\baselineskip} % Whitespace under the publisher logo
	
	 % Publication year
	
	{\large August 15,2017} % Publisher
\end{titlepage}


% \maketitle

% \begin{abstract}

%\end{abstract}

\newpage
\pagenumbering{roman}
\tableofcontents
\newpage



\section{Case Story}
`WannaCry' is a malware which is a scary type of Trojan virus is called ``Ransomware". 
This malware in effect holds the infected computer hostage & demands that the victim 
pay a ransom in order to regain access to the files on his or her computer.

\\
In May 2017, Worldwide Cyberattack by wannacry, where only Microsoft Windows Operating
System's computers are suffered. WannaCry encrypt the infected computer file and demanding
ransom payments in the Bitcoin cryptocurrency. This attack began on Friday, May 2017 and  
within a day was reported to have infected more than 2,30,000 computers in over 150 countries.
Part of the United Kingdom’s {\bf National Health Service (NHS)} was infected where Doctor
appointments and operations were canceled and patients’ lives were at stake. Other major attacks were made on Spain’s
{\bf Telefonica}, {\bf FedEx} and {\bf Deutsche Bahn}.

\\
The initial infection was likely through an exposed vulnerable {\bf SMB(Server Message Block)}  port,
rather than email phishing as initial assumed. Initial stage, the malware executed a computer when it
fail to find out ``kill switch” domain name and encrypts the computer files, then it try to exploit the
SMB vulnerability to spread out to random computers on the Internet and ``laterally" to computers on the same network.
After encrypted data, it give a message to user and demands a payment of around \$300 bitcoin within three days or \$600
bitcoin within six days otherwise after seven days, they will destroy all data. As of 14 June 2017,
totaling \$130,634.77 has been transferred where total of payment 327.

\\
Shortly after the attack began, a young Web Security Researcher from North Devon in England who was named 
Marcus Hutchins then known as {\bf MalwareTech} discovered an effective `Kill switch’ by registering a domain
name which he found in the code of the ransomware.It really slowed the spread of the infection, 
effectively halting the initial outbreak on Monday, 15 May 2017, but new versions have since been detected
that lack the kill switch.

\\
WannaCry malware was first discovered by the United States {\bf National Security Agency (NSA)}.
But they used it to create own offensive work, rather than report it to Microsoft. This flaw and
the tool to exploit it with malicious software were publicized recently by a hacker group by the name `Shadow Brokers’.

\\
In 14 March 2017, Microsoft eventually discovered the vulnerability and they issued security bulletin MS17-010 which detailed 
the flaw and announced that {\bf patches} had been released for all Windows versions that were currently supported at that time.

\\
This Cyber Attack mainly occurs for SMB vulnerability though it began by email phishing. When user updated their computer and 
removed vulnerability then the infected is slowed. If Microsoft knew this vulnerability before this attack then this incident
never happened.

\newpage
\section{Offenses}
List to offenses done by the hacker is given below:
\begin{enumerate}
 \item Hacker has transferred malware by unauthorized email.
 \item Hacker was demanding ransom payments.
 \item Full Operating System hacked by hacker.
 \item Hacker unauthorized access to the computer.
 \item Hacker encrypted all confidential data.
 \item Hacker tampering with computer source code.
\end{enumerate}

\section{Attack on me}
Suppose, I am a Windows 7 Operating System user and I am attack by \textbf{WannaCry}. After attack, When I turn on my computer then it shows me 
a message where was written \textbf{`Your important files are encrypted. If you want to decrypt, you need to pay. \\You only three days
to submit the payments. After that, the price will be double. Also if you don't pay in seven days, you won't be able to recover your file 
forever and you can pay only by \textit{Bitcoin}'}. I didn't find any way to decrypt my data and it has forced to me for paying the ransom.
\section{How to Avoid}
           
           After this attacked, Microsoft give us some guideline for stay safe:
           \begin{itemize}
            \item Be careful NOT to click on harmful links in your emails. 
            \item Try to avoid visiting unsafe or unreliable sites.
            \item Don't click on a untrusted link which takes away to an unsafe website. 
            \item If you receive a message from your friend with a link, ask him before 
            opening the link to confirm, (infected machines send random messages with links).
            \item Keep your files backed up regularly and periodically.
            \item Use anti virus and Always make have the last update.
            \item Make sure your windows have the last update close the gap.
           \end{itemize}


\section{Reference}
\begin{itemize}
 \item URL: https://en.wikipedia.org/wiki/WannaCry\_ransomware\_attack
 \item URL: http://edition.cnn.com/2017/05/14/opinions/wannacrypt-attack
 -should-make-us-wanna-cry-about-vulnerability-urbelis/index.html
 \item URL: https://answers.microsoft.com/en-us/windows/forum/windows\_10
 -security/wanna-cry-ransomware/5afdb045-8f36-4f55-a992-53398d21ed07?auth=1
 \item URL: https://www.cnet.com/news/wannacry-wannacrypt-uiwix-ransomware-everything-you-need-to-know/
\end{itemize}

All website visiting time on August 14, 2017, at 11:30 am to on August 15, 2017, at 9:20 pm. 


\end{document}
